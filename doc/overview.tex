\documentclass[12pt,a4paper,notitlepage]{article}
\usepackage[utf8]{inputenc}
\usepackage[a4paper,textwidth=17cm, top=2cm, bottom=3.5cm]{geometry}
\usepackage[T1]{fontenc}
\usepackage{natbib}
\usepackage{ngerman} 
\usepackage{amssymb,amsmath}
\usepackage{listings}
\usepackage{eurosym}
\usepackage{wasysym}
\usepackage{amsthm}
\usepackage{tabularx}
\usepackage{multirow}
\usepackage{multicol}
\usepackage{hyperref}
\usepackage{tabularx}
\usepackage{commath}
\usepackage{subfigure}
\usepackage[pdftex]{graphicx,color}
\usepackage{epstopdf}
\newcommand{\re}{\text{Re}}
\newcommand{\im}{\text{Im}}
\newcommand{\foonote}[1]{\footnote{#1}}
\newcommand{\degree}{\ensuremath{^\circ}}
\author{Sebastian Götte und Matti Möll}
\title{Entwicklung eines Datenbanksynchronisations-Mikroframeworks auf Basis konvergenter, replizierter Datentypen}
\date{27.10.2014}
\begin{document}
\maketitle

\section{History of CDRTs and the current state of research}
Data synchronisation has been an important area of research for several decades. Recently, discourse has been heightened
due to increasing parallelism of contemporary system architectures. Traditionally, such systems have either relied on a
central authority\footnote{See every RDBMS in existence, OpenLDAP and ActiveDirectory, RSS and apt among others} or been
largely limited to read-only affairs\footnote{See e.g.\ BitTorrent and other classical file sharing systems}.

Well-defined automatic merge strategies to reconcile diverging realities in different network locations have been in
devlopment for some time now, especially under the aegis of distributed version control systems (DVCSs) such as git,
Mercurial and Darcs but have not yet seen widespread deployment.

One possible solution to this problem is what we call Convergent Replicated Data Types (CRDTs). CRDTs have been used by
computer scientists for at least 40 years. The concept was only formalized and its name coined five years ago in
\cite{inria09}. Since then, there has been some further research into the formal properties of CRDTs\footnote{See
\cite{inria11}} and first real-world implementations of the concept using the term \em{CRDT} have appeared\footnote{See
e.g.\ \cite{riak} resp.\ \cite{riak20-announcement} and \cite{riak-crdt} and \cite{roshi}}.

In 2007, Amazon.com published a paper detailing the architecture of their \em{dynamo} key-value store. From a technical
standpoint, Dynamo's replication system is behaving like a CRDT set implementation.
%FIXME

We want to give an overview of the history and existing implementations of CRDTs followed by an overview of some
concepts that may be used in conjunction with CRDTs to provide useful higher-level behavior and end with some exemplary
possible real-world use cases.

\section{Convergent replication}
CRDTs have first been defined in \cite{inria09}. \cite{inria09} are using the term \em{CRDT} for \em{Collision-free
Replicated Data Type} and are distinguishing between \em{CvRDTs} (\em{Convergent Replicated Data Types}) which they are
also calling \em{State-based CRDTs}, and \em{CmRDTs} (\em{Commutative Replicated Data Types}) which they are also
calling \em{Operation-based} or \em{Op-based CRDTs}. Since both concepts are shown to be exactly equivalent, we are
using the acronym \em{CRDT} for either approach and settled on the long form \em{Convergent Replicated Data Type} since
it is describing both very well and is sufficiently handy.

\subsection{Operation-based CRDTs}
An op-based CRDT is a data type whose value is defined solely by a set of commutative operations applied to a common
base state. For illustration, consider an up/down counter as a simple example. The initial state would be $0$ and the
operations would be to add or subtract a number. The current counter value is defined as the sum of all add/subtract
operations performed on this instance so far. Since addition is commutative, the order of these operations does not
matter. In a distributed system, additions and subtractions on the same counter can be performed simultaneously on
multiple nodes, and the resulting conflict can be resolved by each node telling each other node all operations that have
been performed on its instance.

\subsection{State-based CRDTs}
A state-based CRDT is a data type which is associated with a partial ordering on its value space. A simple example for a
state-based CRDT is an add-only set. The associated partial ordering is the subset relation. The merge operation is the
set union.

State- and Operation-based CRDTs are equivalent in that each can be used to emulate the other. Most existing
implementations use a state-based storage, where some do incorporate some op-based-like behavior %FIXME
for more efficient sync\footnote{e.g.\ Amazon Dynamo is computing a delta using a Merkle tree and then only transmitting
the subset of changed entities, which is equivalent to the set of operations since last launch.}.

\section{Possible directions for research}

\section{Possible use cases}

\bibliographystyle{te}
\bibliography{overview}

\end{document}
